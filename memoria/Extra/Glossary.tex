\newacronym{APK}{APKs}{Android Application Package}
\newacronym{NFC}{NFC}{Near Field Communication}
\newacronym{HTTP}{HTTP}{Hypertext Transfer Protocol}
\newacronym[plural=URLs]{URL}{URL}{Uniform Resource Locator}
\newacronym[plural=BBDDs,longplural=Bases de datos]{BBDD}{BBDD}{Base de datos}
\newacronym[plural=SPAs]{SPA}{SPA}{Single Page Application}
\newacronym{IP}{IP}{Internet Protocol}
\newacronym{MV*}{MV*}{Model-View-Whatever}
\newacronym{MVVP}{MVVP}{Model-View-ViewModel}
\newacronym{IAM}{IAM}{Identify and Access Management}
\newacronym{ORM}{ORM}{Object Relational Mapping}
\newacronym{JSON}{JSON}{JavaScript Object Notation}
\newacronym{CSS}{CSS}{Cascading Style Sheets}
\newacronym{CSS3}{CSS3}{Cascading Style Sheets v3}
\newacronym{Sass}{Sass}{Syntactically Awesome Stylesheets}
\newacronym{MVC}{MVC}{Model-View-Controler}
\newacronym{GPS}{GPS}{Global Positioning System}
\newacronym{DI}{DI}{Dependency Injector}
\newacronym{SQL}{SQL}{Structure Query Lenguage}
\newacronym{PFC}{PFC}{Proyecto Fin de Carrera}
\newacronym{HTML}{HTML}{HyperText Markup Language}
\newacronym{npm}{npm}{The Node Package Manager}
\newacronym{LTS}{LTS}{Long Term Support}
\newacronym{CLI}{CLI}{Command Line Interface}
\newacronym{JS}{JS}{JavaScript}
\newacronym{ADB}{ADB}{Android Debugger Bridge}
\newacronym[plural=AVDs, longplural=Android Virual Devices]{AVD}{AVD}{Android Virual Device}
\newacronym{SDK}{SDK}{Software Developer Kit}
\newacronym{PC}{PC}{Personal Computer}
\newacronym{IDE}{IDE}{Integrated Development Environment}
\newacronym{NoSQL}{NoSQL}{NoSQL}
\newacronym{USB}{USB}{Universal Serial Bus}
\newglossaryentry{AOSP}{
first={Android Open Source Project},
name={AOSP},
description={Versión básica del sistema operativo para smartphone de Google, Android.},
sort={AOSP}
}
\newglossaryentry{JDK}{
first={Java Development Kit},
name={JDK},
description={Conjunto oficial de herramientas necesarias para el desarrollo de aplicaciones en Java.},
sort={JDK}
}
\newglossaryentry{JRE}{
first={Java Runtime Enviroment},
name={JRE},
description={Entorno de ejecución para software desarrollado en Java.},
sort={JRE}
}
\newglossaryentry{MEAN}{
first={MongDB-ExpressJS-Angular-NodeJS},
name={MEAN},
description={Conjunto de herramientas para la programación de aplicaciones distribuidas usando JavaScript como lenguaje en todas las fases. El nombre proviene de las cuatro herramientas principales usadas en este conjunto (MongDB-ExpressJS-Angular-NodeJS)},
sort={MEAN}
}
\newglossaryentry{API}{
first={Application Programming Interface},
name={API},
plural={APIs},
description={La interfaz de Programación de Aplicaciones  es el conjunto de funciones y procedimientos (o métodos, en la programación orientada a objetos) que ofrece cierta biblioteca para ser utilizado por otro software como una capa de abstracción. },
sort={API}
}
\newglossaryentry{DOM}{
first={Document Object Model},
name={DOM},
description={El Modelo de Objetos para Representación de Documentos se trata de una interfaz de plataforma que proporciona un conjunto estándar de objetos para representar documentos HTML y XML, un modelo estándar sobre cómo pueden combinarse dichos objetos, y una interfaz estándar para acceder a ellos y manipularlos. },
sort={DOM}
}
\newglossaryentry{XPath}{
first={XML Path Language},
name={XML Path Language},
description={XPath es un lenguaje que permite construir expresiones que recorren y procesan un documento XML. },
sort={XPath}
}\newglossaryentry{AJAX}{
first={AJAX},
name={Asynchronous JavaScript And XML},
description={AJAX es una técnica de desarrollo web para crear aplicaciones interactivas o RIA (Rich Internet Applications). Estas aplicaciones se ejecutan en el cliente, es decir, en el navegador de los usuarios mientras se mantiene la comunicación asíncrona con el servidor en segundo plano. De esta forma es posible realizar cambios sobre las páginas sin necesidad de recargarlas, mejorando la interactividad, velocidad y usabilidad en las aplicaciones. },
sort={AJAX}
}\newglossaryentry{mixin}{
first={mixin},
name={mixin},
plural={mixins},
description={En la programación orientada a objetos, una clase mixin es una clase que contiene propiedades y métodos que pueden ser reutilizados por otras clases sin necesidad de tener una relación de parentesco con ellas. En el caso de SASS, permiten definir estilos que se pueden reutilizar y parametrizar. },
sort={mixin}
}
\newglossaryentry{SEMVER}{
first={SEMVER},
name={SEMVER},
plural={SEMVER},
description={El sistema SEMVER es un conjunto de reglas para proporcionar un significado claro y definido a las versiones de proyecto de software. El sistema SEMVER se compone de 3 números, siguiendo la estructura X.Y.Z, donde X (major) indica un cambio rupturista, Y (minor) indica cambios compatibles con versiones anteriores y Z (path) indica resoluciones de bugs. },
sort={SEMVER}
}
