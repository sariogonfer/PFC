\chapter*{Trabajos futuros}
\addcontentsline{toc}{chapter}{Trabajos futuros}
\markboth{TRABAJOS FUTUROS}{TRABAJOS FUTUROS}

En este \gls{PFC} se ha visto los rasgos más importantes de la tecnología utilizada en aplicaciones híbridas y se ha profundizado en el framework Ionic2, y por tanto, también en Angular2. Lo visto en esta memoria puede servir como base para la realización de aplicaciones usando estas tecnologías.

Tanto Ionic2 como Angular2, son frameworks de gran tamaño con multitud de módulos y características que no se han tratado en esta memoria y que podrían ser usados para futuras prácticas. Una muy interesante podría ser la creación de una aplicación híbrida que actuase como cliente de una aplicación servidora, usando el stack \gls{MEAN} en las diferentes partes del sistema. Este sistema podría consistir únicamente en una gestión de usuarios y perfiles muy simple pero que permitiera mostrar el esqueleto de un sistema de estas características y las herramientas que se ven involucradas, sirviendo de base para la creación de sistemas más complejos.

Si nos centramos solo en el lado del cliente, dentro de Ionic Native nos ofrece diferentes módulos con los que usar las distintas características del dispositivo, como por ejemplo la cámara, con las cuales poder añadir nuevas funciones a las prácticas ya mostradas (incluir los sensores de movimiento en la práctica del paisaje, o el sensor \gls{NFC} en la del mapa de recordatorios) o crear aplicaciones nuevas que aprovechen la potencia de estas.

Por último, hacer el mismo ejercicio que se ha hecho utilizando Ionic2, pero usando alguno de los otros frameworks comentados en la introducción (Rect Native, Native Script, \ldots) lo que permitirá comparar diferentes tecnologías permitiendo al lector elegir cual es la que más le conviene.
