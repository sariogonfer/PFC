\chapter*{Resumen}
\addcontentsline{toc}{chapter}{Resumen}
\markboth{RESUMEN}{RESUMEN}

En los últimos tiempos los smartphones han experimentado un gran crecimiento tanto en el número de terminales en el mercado como en la variedad de estos. Junto con este crecimiento de hardware, se ha experimentado un aumento igual de significativo en el software que aprovecha las capacidades de estos dispositivos. A la hora de crear este software, los desarrolladores se encuentran con la falta de homogeneidad que encontramos en este tipo de dispositivos (diferentes fabricantes, diferentes plataformas, hardware con distintas capacidades, \ldots).

En este PFC trataremos desde un punto de vista teórico y práctico el uso de Ionic2 en la realización de aplicaciones híbridas para smartphones. Este tipo de aplicaciones intentan que el desarrollo de una aplicación orientada a smartphones pueda servir para los distintos dispositivos en el mercado sin importar la plataforma que ejecuten o el hardware que montan. En el caso de Ionic2, esto se consigue programando la aplicación utilizando tecnología web (HTML, JS y CSS) y ejecutándola sobre un contenedor nativo. Ionic2 se encarga de ofrecer las herramienta necesarias para facilitar este desarrollo, encargándose de compilar la aplicación, generar el contenedor, ofrecer una API unificada sin importar la plataforma, \ldots

A lo largo de esta memoria se hará una introducción a las tecnologías relacionadas con este tipo de aplicaciones, para más tarde, proponer una serie de prácticas con las que el lector obtendrá los conocimientos necesarios para la realización de una aplicación utilizando Ionic2.
