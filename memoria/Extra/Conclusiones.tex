\chapter*{Conclusiones}
\addcontentsline{toc}{chapter}{Conclusiones}
\markboth{CONCLUSIONES}{CONCLUSIONES}

Las aplicaciones móviles se han convertido en uno de los sectores más importantes dentro del mundo tecnológico, muestra de ello es la gran comunidad de desarrolladores que existe alrededor de las tecnologías involucradas. Este crecimiento también se ha visto reflejado en las herramientas que disponemos para desarrollar estas aplicaciones. Resulta curioso ver como la gran mayoría de estas herramientas son libres, aún siendo creadas y mantenidas por grandes empresas. Esto es de agradecer, ya que nos permite tener un mayor número de opciones y elegir la que mejor se adapte a nuestras necesidades sin tener que preocuparnos en una posible inversión.

Por otro lado, también hay que destacar la evolución que han sufrido las tecnologías web en estos años, ya sea a través de nuevos estándares, frameworks, lenguajes, \ldots Por un lado ha permitido mejorar los diseños de las páginas, haciéndolos más atractivos para el usuario, y facilitando el trabajo del diseñador. En este sentido, \gls{CSS3} y sus animaciones, diferentes frameworks como Bootstrap o la definición de nuevas propiedades como los Flex-box han sido los responsables. Por otro, y quizás donde más se ha avanzado, es la parte lógica de las aplicaciones. Se ha pasado de estar limitados por un lenguaje como JavaScript, el cuál no está pensado para aplicaciones complejas, a poder trabajar con características de lenguajes más avanzados (clases, herencias, tipado, \ldots) y poder utilizar patrones de diseños. Todo esto ha hecho que JavaScript se convierta en una alternativa muy valida no solo para su uso en front-end, si no también para utilizarlo en back-end. Destacar el papel de NodeJS en esta evolución.

En cuanto a las aplicaciones híbridas, el tema sobre el que gira este \gls{PFC}, han encontrado su hueco al calor de la batalla que existe entre las dos grandes plataformas móviles, Android e iOS; y la variedad de dispositivos en un mercado donde a la lucha entre los fabricantes tradicionales se han unido nuevos originarios de China. Estas aplicaciones se han aprovechado de los avances de las tecnologías web comentadas previamente y han ido evolucionando cada vez más convirtiéndose en una alternativa seria a las aplicaciones nativas. Gracias a este tipo de aplicaciones, y la cantidad de herramientas que han surgido, no hace falta tener unos conocimientos avanzados para crear tu propia aplicación.

Pero no todo son bondades en cuanto a las aplicaciones híbridas, y aún tienen ciertos aspectos que mejorar. Es por ello que cuando se trata de aplicaciones realmente complejas, que requieran de un diseño cuidado o que tengan ciertos requerimientos, por ejemplo, de seguridad, las aplicaciones nativas están un paso por delante siendo la mejor opción siempre que se tengan los recursos necesarios.

Por último añadir que la practicas propuestas en esta memoria podrá ayudar a los estudiantes a entrar en el mundo de la programación aplicaciones hibridas, ya que se empieza desde lo más básico y se va aumentando la dificultad de las aplicaciones gradualmente. Además, se intenta ver el mayor número de características de las tecnologías usadas, dejando en manos de los estudiantes el profundizar más ellas según sus necesidades.
