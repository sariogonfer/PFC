\clearpage
\chapter{Prácticas} \label{ch:practices}

Una máxima en el mundo de la programación es: \emph{"La mejor forma de aprender a programar es programando"}. En este capítulo se van a proponer una serie de prácticas con los que poder ver Ionic en acción. Para la realización de estas aplicaciones, daremos por sentado que se dispone de un ordenador con Ionic2 instalado (ver anexo \nameref{ch:ionic}) y un editor de código, en nuestro caso utilizaremos \nameref{ch:atom}. Se recomienda crear un nuevo directorio que sirva como área de trabajo en el que guardaremos los proyectos.

Como se suele hacer en cualquier lenguaje de programación, comenzaremos programando un \emph{Hello World!}, esto es, una aplicación sencilla que lo único que haga sea imprimir el mensaje ``Hello World!'' en el dispositivo de salida. Esta práctica servirá de toma de contacto con la estructura de las aplicaciones construidas con Ionic. A continuación veremos como depurar nuestro desarrollo utilizando un emulador o directamente sobre un dispositivo real y como construir nuestra aplicación. Con esta base, continuaremos viendo aspecto más específico de sobre la tecnología, como el \emph{Data Binding}, los servicios, las animaciones, el uso de la \gls{API} del sistema, \ldots a través de tres prácticas es las que programaremos: un \textbf{Cronómetro} \ref{sec:crono}, un \textbf{Paisaje} \ref{sec:paisaje} y una aplicación que gestione \textbf{Recordatorios asociados a una localización} \ref{sec:recordatorios}.

Se dará por supuesto que el lector de este documento cuenta con unos conocimientos básicos de los lenguajes JavaScript, \gls{HTML} y \gls{CSS}, por lo que no se profundizará en conceptos referentes a estas tecnologías si no es estrictamente necesario.
